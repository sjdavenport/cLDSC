Let $ Y = X\beta + \epsilon$, where $ Y \in \mathbb{R}^n$ is a vector of phenotypes, $ X \in \mathbb{R}^{n\times m}$ is the matrix of genotypes normalized to be mean zero and variance one 1, $ \beta \in \mathbb{R}^m$ of effect sizes and $ \epsilon \in \mathbb{R}^n$ is the error. In this model we assume that $ \mathbb{E}(\epsilon) = 0  $, $ \var(\epsilon) = (1-h^2)I $, $ \mathbb{E}(\beta)  = 0$ and $  \var(\beta) = (h^2/M)I $, where $ h^2 $ is the heritability. This is the same setting as in  \cite{Bulik2015}, see e.g. the beginning of their supplementary material.\\

\noindent For $ 1\leq j \leq m $, let $ X_j $ be the $ j $th column of $ X $ and let $ \hat{\beta}_j = X_j^TY/n $ and set $ u_j = n\hat{\beta}_j^2 $ be the $ \chi^2 $ statistics. Moreover let
\begin{equation*}
	l_j = \sum_{k = 1}^m r^2_{jk}  = \sum_{k = 1}^m \mathbb{E}(X_{1j}X_{1k})
\end{equation*}
be the true LD scores. And let 
\begin{equation*}
	\hat{l}_j = X_j^TXX^TX_j/n^2 
\end{equation*}
be the estimates of the LD scores from the data. Since these may not be directly recorded let $ \tilde{l}_j $ be an estimate of $ l_j $ from an independent reference dataset. 

Then as I understand LD score regression fits the linear model (up to regression weightings),
\begin{equation*}
	u_j = a + \frac{n}{m}\tilde{l}_j h^2 + \eta,
\end{equation*}
where $ a $ represents the intercept term, and $ \eta $ the noise. This performing linear regression results in estimates $ \hat{a} $ and $ \hat{h}^2$ for the intercept and the heritability. Here importantly the estimates of the LD scores from the reference dataset are used instead of the actual values $ (l_j)_{j = 1}^m $ since these are unknown. Running this regression seems strange to me since it is based on the approximation
\begin{equation*}
	\mathbb{E}(u_j) \approx\frac{n}{m}l_j h^2 + na + 1
\end{equation*}
that they derive in their paper. However because all of the $ u_j $s share the same $ X $ to me it doesn't seem possible to use them to infer on $ \mathbb{E}(u_j) $ which is the expectation of $ u_j $ given that $ X $ can vary randomly. I.e. I would have thought you would need to have samples from the $ u_j $ distribution which had a different original $ X $ in order to be able to infer on $ \mathbb{E}(u_j) $.
% My prior would be that using the resulting estimates would as such not be a consistent estimator for $ h^2 $, even if the $ l_j $s were known and the regression were against them instead of the $ \tilde{l}_j $s. I would like to understand whether that intuition is incorrect. 

However I think that it would instead be possible to use the $ u_j $ to infer on $ \mathbb{E}(u_j|X) $ since they share the same $ X $. In particular the derivation in the supplementary of \cite{Bulik2015} implies that 
\begin{align*}
	 \mathbb{E}(u_j|X) = n\var(\hat{\beta}_j| X) = \frac{nh^2}{m}\hat{l}_j + 1-h^2= h^2\left(\frac{n}{m}\hat{l}_j - 1\right) + 1
\end{align*} 
Note that since the only dependence on $ X $ in this expression is via the $ \hat{l}_j $  in fact $ \mathbb{E}(u_j|X) = \mathbb{E}(u_j| \hat{l}_1, \dots, \hat{l}_m) = \mathbb{E}(u_j|\hat{l}_j) $.\\

\noindent If the $ \hat{l}_j $s were known it would thus make sense to instead run the regression
\begin{equation*}
	u_j = h^2\left(\frac{n}{m}\hat{l}_j - 1\right) + 1 + \eta
\end{equation*}
with a fixed intercept of $ 1 $ and noise error term $ \eta $ and solve to obtain an estimate of $ h^2$ (also adjusting using regression weights to account for the dependence over $ j $). I had thought (prior to your email) that, even though $ X  $ was not known, the values $ \hat{l}_j $ were stored. In fact that does not seem to be true which is a shame. Instead though I would propose to run the regression 
\begin{equation*}
u_j = h^2\left(\frac{n}{m}\tilde{l}_j - 1\right) + 1 + \eta
\end{equation*} 
and solve for $ h^2. $ I would have thought a priori that this would do a better job than LD score regression because it tries to target $ \mathbb{E}(u_j|X) $ rather than $ \mathbb{E}(u_j) $. But if not I would like to understand what is better about LD score regression compared to this approach.

%Worth noting that this approach is asymptotically equivalent to Armin's GWASH estimator, though it has the advantage that it only depends on estimating a single parameter from the reference sample. In particular one needs to calculate $ \frac{1}{m}\sum_{j = 1}^m\left(\frac{n}{m}\hat{l}_j - 1\right) $, which can be estimated by $ \frac{1}{m}\sum_{j = 1}^m\left(\frac{n}{m}\tilde{l}_j - 1\right) $.
