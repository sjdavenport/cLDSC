\section{Conditional vs unconditional regression}
Let $ Y = X\beta + \epsilon$, where $ Y \in \mathbb{R}^n$ is a vector of phenotypes, $ X \in \mathbb{R}^{n\times m}$ is the matrix of genotypes normalized to be mean zero and variance one 1, $ \beta \in \mathbb{R}^m$ of effect sizes and $ \epsilon \in \mathbb{R}^n$ is the error. In this model we assume that $ \mathbb{E}(\epsilon) = 0  $, $ \var(\epsilon) = (1-h^2)I $, $ \mathbb{E}(\beta)  = 0$ and $  \var(\beta) = (h^2/m)I $, where $ h^2 $ is the heritability. This is the same setting as in  \cite{Bulik2015}, see e.g. the beginning of their supplementary material.\\

\noindent For $ 1\leq j \leq m $, let $ X_j $ be the $ j $th column of $ X $ and let $ \hat{\beta}_j = X_j^TY/n $ and set $ u_j = n\hat{\beta}_j^2 $ be the $ \chi^2 $ statistics. Moreover let
\begin{equation*}
	l_j = \sum_{k = 1}^m r^2_{jk}  = \sum_{k = 1}^m \mathbb{E}(X_{1j}X_{1k})^2
\end{equation*}
be the true LD scores, where $ r_{jk} = \mathbb{E}(X_{1j}X_{1k}) $. Define $ \hat{r}^2_{jk} = \frac{1}{n}\sum_{i = 1}^n X_{ij}X_{ik} $. Then let
\begin{equation*}
	\hat{l}_j = X_j^TXX^TX_j/n^2 = \sum_{k = 1}^m \hat{r}^2_{jk} 
\end{equation*}
be the estimates of the LD scores from the covariate matrix $ X $ corresponding to the original dataset. Since these may not be directly recorded let $ \tilde{l}_j $ be an estimate of $ l_j $ from an independent reference dataset. 

Then as I understand LD score regression fits the linear model (up to regression weightings),
\begin{equation*}
	u_j = a + \frac{n}{m}\tilde{l}_j h^2 + \eta,
\end{equation*}
where $ a $ represents the intercept term, and $ \eta $ the noise. This performing linear regression results in estimates $ \hat{a} $ and $ \hat{h}^2$ for the intercept and the heritability. Here importantly the estimates of the LD scores from the reference dataset are used instead of the actual values $ (l_j)_{j = 1}^m $ since these are unknown. Running this regression seems strange to me since it is based on the approximation
\begin{equation*}
	\mathbb{E}(u_j) \approx\frac{n}{m}l_j h^2 + na + 1
\end{equation*}
that they derive in their paper. However because all of the $ u_j $s share the same $ X $ to me it doesn't seem possible to use them to infer on $ \mathbb{E}(u_j) $ which is the expectation of $ u_j $ given that $ X $ can vary randomly. I.e. I would have thought you would need to have samples from the $ u_j $ distribution which had a different original $ X $ in order to be able to infer on $ \mathbb{E}(u_j) $.
% My prior would be that using the resulting estimates would as such not be a consistent estimator for $ h^2 $, even if the $ l_j $s were known and the regression were against them instead of the $ \tilde{l}_j $s. I would like to understand whether that intuition is incorrect. 

However I think that it would instead be possible to use the $ u_j $ to infer on $ \mathbb{E}(u_j|X) $ since they share the same $ X $. In particular the derivation in the supplementary of \cite{Bulik2015} implies that 
\begin{align*}
	 \mathbb{E}(u_j|X) = n\var(\hat{\beta}_j| X) = \frac{nh^2}{m}\hat{l}_j + 1-h^2= h^2\left(\frac{n}{m}\hat{l}_j - 1\right) + 1
\end{align*} 
Note that since the only dependence on $ X $ in this expression is via the $ \hat{l}_j $  in fact $ \mathbb{E}(u_j|X) = \mathbb{E}(u_j| \hat{l}_1, \dots, \hat{l}_m) = \mathbb{E}(u_j|\hat{l}_j) $.\\

\noindent If the $ \hat{l}_j $s were known it would thus make sense to instead run the regression
\begin{equation*}
	u_j = h^2\left(\frac{n}{m}\hat{l}_j - 1\right) + 1 + \eta
\end{equation*}
with a fixed intercept of $ 1 $ and noise error term $ \eta $ and solve to obtain an estimate of $ h^2$ (also adjusting using regression weights to account for the dependence over $ j $). I had thought (prior to your email) that, even though $ X  $ was not known, the values $ \hat{l}_j $ were stored. In fact that does not seem to be true which is a shame. Instead though I would propose to run the regression 
\begin{equation*}
u_j = h^2\left(\frac{n}{m}\tilde{l}_j - 1\right) + 1 + \eta
\end{equation*} 
and solve for $ h^2. $ I would have thought a priori that this would do a better job than LD score regression because it tries to target $ \mathbb{E}(u_j|X) $ rather than $ \mathbb{E}(u_j) $. But if not I would like to understand what is better about LD score regression compared to this approach.

\section{Convergence of the LD scores}
Applying the CLT to the correlation coefficients, we obtain the following lemma.
\begin{lemma}
	Suppose that the rows of $ X $ are i.i.d. and standardized, then 
	\begin{equation*}
		\frac{n\left(\hat{r}_{jk} - r_{jk}\right)^2}{\var(X_{1j}X_{1k})} \convd \chi^2_1.
	\end{equation*}
\end{lemma}
\begin{proof}
	By the Lindeberg CLT, since $ \hat{r}_{jk} = \frac{1}{n}\sum_{i = 1}^n X_{ij}X_{ik}, $
	\begin{equation*}
		\sqrt{n}(\hat{r}_{jk} - r_{jk}) \convd N(0, \var(X_{1j}X_{1k}) ).
	\end{equation*}
	Need to assume that the Lindeberg conditions hold on the products and basically check when that is reasonable. the boundedness of the Xs (before standardization of them will probably be helpful here)
\end{proof}

\begin{lemma}
	Let $ \sigma_j $ be the standard deviation of $ (X_{ij})_{1\leq i \leq n} $ before standardization. Then $ X_j = (\tilde{X}_j - \hat{\mu}_j)/\hat{\sigma_j},$ where $ \tilde{X} $ is the original matrix of genotypes (i.e. unstandardized and not demeaned.) Then $ \hat{r}_{jk} $ 
\end{lemma}

\begin{proposition}
	Suppose that $ m/n \rightarrow c \in \mathbb{R} $. Suppose further that given $ j \in \lbrace 1, \dots, m\rbrace $, there exists a neighbourhood $ N(j) \subset \lbrace 1, \dots, m\rbrace $ such that $ |N(j)| = o(m) $ and such that $ X_j $ is independent of $ X_k $ for all $ k \not\in N(j) $. Then,
	\begin{equation*}
		\hat{l}_j \convd N\left(\sum_{k \in N(j)} r^2_{jk} + c, \right).
	\end{equation*}
\end{proposition}
\begin{proof}
	The neighbourhood independence property implies that for $ k \not\in N(j), r_{jk} = 0 $. As such
	\begin{align*}
		\hat{l}_j = \sum_{k = 1}^m \hat{r}^2_{jk} = \sum_{k \in N(j)} \hat{r}^2_{jk} + \sum_{k \not\in N(j)} \hat{r}^2_{jk} =  \sum_{k \in N(j)} \hat{r}^2_{jk} + \frac{c}{m}\sum_{k \not\in N(j)} n\hat{r}^2_{jk}
	\end{align*}
	Conditional on $ X_j $, $ \hat{r}_{jk} $ and $ \hat{r}_{jl} $ are independent for $ l $ and $ k $ sufficiently far apart! Moreover, using the approximation in the supplementary of \cite{Bulik2015}, it follows that $ \mathbb{E}(\hat{r}^2_{jk}) = \frac{1}{n} + O(n^{-2}) $
	
	In order to apply the Lindeberg condition, first note that 
	
	Now we can write 
	\begin{equation*}
		\frac{c}{m}\sum_{k \not\in N(j)} n\hat{r}^2_{jk} = 	\frac{c}{m}\sum_{k \not\in N(j)} n\left(\hat{r}^2_{jk} - \mathbb{E}(\hat{r}^2_{jk})\right) + \frac{c}{m}\sum_{k \not\in N(j)} \left(1 + O(n^{-1})\right)
	\end{equation*}
	
	Now 
	\begin{equation*}
	\var\left(\frac{c}{m}\sum_{k \not\in N(j)} n\hat{r}^2_{jk} | X_j\right) = \sum_l \sum_k 
	\end{equation*}
	
	
\end{proof}

When using a reference sample in which the ratio $ m/n $ is different, this should be corrected for and taken into account in the estimation. Though this effect will be relatively negligible!

\section{Morphometricity}
The advantage of fixing u and making Z random is that e.g. in your simulations one has to guess the distribution of u which might not be true in practice whereas we observe Z and thus have a lot of information about its distribution.

%Worth noting that this approach is asymptotically equivalent to Armin's GWASH estimator, though it has the advantage that it only depends on estimating a single parameter from the reference sample. In particular one needs to calculate $ \frac{1}{m}\sum_{j = 1}^m\left(\frac{n}{m}\hat{l}_j - 1\right) $, which can be estimated by $ \frac{1}{m}\sum_{j = 1}^m\left(\frac{n}{m}\tilde{l}_j - 1\right) $.
